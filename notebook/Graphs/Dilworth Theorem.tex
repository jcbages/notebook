An antichain in a partially ordered set is a set 
of elements no two of which are comparable to each 
other, and a chain is a set of elements every two 
of which are comparable. Dilworth's theorem states 
that there exists an antichain A, and a partition 
of the order into a family P of chains, such that 
the number of chains in the partition equals the 
cardinality of A. When this occurs, A must be the 
largest antichain in the order, for any antichain 
can have at most one element from each member of 
P. Similarly, P must be the smallest family of 
chains into which the order can be partitioned, 
for any partition into chains must have at least 
one chain per element of A. The width of the 
partial order is defined as the common size of A and P.

An equivalent way of stating Dilworth's theorem is that, 
in any finite partially ordered set, the maximum number 
of elements in any antichain equals the minimum number 
of chains in any partition of the set into chains. A 
version of the theorem for infinite partially ordered 
sets states that, in this case, a partially ordered 
set has finite width w if and only if it may be 
partitioned into w chains, but not less.